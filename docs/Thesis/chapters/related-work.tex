%!TEX root = ../main.tex

\chapter{Related Work}\label{ch:related-work}

\section{Automatic Object Placement}

In recent years, there have been many different
approaches to solve the problem of automatic object placement in virtual
environments. Some of these approaches include rule-based methods,
optimization-based techniques, physics based approaches and machine learning-based algorithms. 
Answer Set Programming (ASP) is a relatively new declarative programming paradigm
that has shown promise in solving combinatorial problems, including object
placement.

This chapter will provide an overview of the existing literature related to
automatic object placement in virtual environments. It will cover various
approaches and techniques used by researchers in this field, highlighting the
strengths and weaknesses of each method. The chapter will also provide a
detailed discussion of the current state of the art in automatic object
placement using ASP and how it can be applied to develop virtual background
scenes. Finally, it will identify the research gaps and open problems that
need to be addressed to further advance this field.

\cite{Shinya_1995} -- Geometric placment with constraints
\cite{Seversky_2006} -- 3D Scene Generation from natural language

\cite{Smith_2011} -- ASP in procedural content generation
